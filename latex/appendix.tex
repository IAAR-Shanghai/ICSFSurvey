\section{Notations} \label{apdx:notation}

\begin{table}[h!]
\caption{Common Notations} \label{tab:symbols}
\centering
\begin{tabular}{p{1.2cm}p{6.5cm}}
\toprule
Symbol & Description \\ 
\midrule
$\boldsymbol{x}$ & Query \\ 
$\mathcal{M}, \mathcal{N}$ & LLMs \\ 
$e$ & Expression type, $e \in \{\text{response}, \text{decoding},\text{latent}\}$ \\
$O_e(\mathcal{M}, \boldsymbol{x})$ & Sampling distribution \\
$\mathcal{Y}$ & Sampling set \\ 
$\boldsymbol{y}_i$ & The $i$-th element in the sampling set \\ 
$\boldsymbol{y}_{0:i}$ & Elements from $0$ to $i$ in the sampling set \\ 
$y^t$ & The $t$-th token in text $\boldsymbol{y}$ \\ 
$f$ & Consistency Signal of Self-Feedback \\ 
$P(\boldsymbol{y}|\boldsymbol{x};\theta)$ & Probability that model parameterized by $\theta$ outputs $\boldsymbol{y}$ given input $\boldsymbol{x}$ \\ 
\bottomrule
\end{tabular}
\end{table}

\section{The Five Responses of GPT-4o} \label{apdx:GPT4o}

\noindent Fig.~\ref{fig:response1_from_gpt4o}, ~\ref{fig:response2_from_gpt4o}, ~\ref{fig:response3_from_gpt4o}, ~\ref{fig:response4_from_gpt4o}, and~\ref{fig:response5_from_gpt4o} show five different responses obtained when querying GPT-4o: How many full stops (periods) are there: ``.!..!..!''. The experiments were conducted on June 14, 2024, accessed via the official website\footnote{\url{https://chat.openai.com/}}.

\begin{figure}[h!]
\small
\centering
\begin{tcolorbox}[colback=blue!5!white,colframe=blue!75!black,title=GPT-4o Response 1,fontupper=\footnotesize,fonttitle=\scriptsize]
To determine the number of full stops (periods) in the string ".!..!..!", we need to count the occurrences of the character '.'. \\
Let's break down the string: \\
1. ".!": 1 full stop \\
2. "..!": 2 full stops \\
3. ".!": 1 full stop \\
Summing these up: \\
1 + 2 + 1 = 4 \\
Therefore, there are 4 full stops in the string ".!..!..!".
\end{tcolorbox}
\caption{GPT-4o Response 1}
\label{fig:response1_from_gpt4o}
\end{figure}

\begin{figure}[h!]
\small
\centering
\begin{tcolorbox}[colback=blue!5!white,colframe=blue!75!black,title=GPT-4o Response 2,fontupper=\footnotesize,fonttitle=\scriptsize]
The sequence ".!..!..!" contains three instances of "." (full stops or periods). Each segment ".!" has one full stop, and there are three such segments.
\end{tcolorbox}
\caption{GPT-4o Response 2}
\label{fig:response2_from_gpt4o}
\end{figure}

\begin{figure}[h!]
\small
\centering
\begin{tcolorbox}[colback=blue!5!white,colframe=blue!75!black,title=GPT-4o Response 3,fontupper=\footnotesize,fonttitle=\scriptsize]
The string ".!..!..!" contains a total of **3** full stops (periods).
\end{tcolorbox}
\caption{GPT-4o Response 3}
\label{fig:response3_from_gpt4o}
\end{figure}



\begin{figure}[h!]
\small
\centering
\begin{tcolorbox}[colback=blue!5!white,colframe=blue!75!black,title=GPT-4o Response 4,fontupper=\footnotesize,fonttitle=\scriptsize]
The string ".!..!..!" contains full stops (periods). To determine the number of full stops in this string, we simply count the occurrences of the period character ('.'). Let's analyze the string: \\
- The first character is a period. \\
- The second character is an exclamation mark. \\
- The third character is a period. \\
- The fourth character is an exclamation mark. \\
- The fifth character is a period. \\
- The sixth character is an exclamation mark. \\
- The seventh character is a period. \\
- The eighth character is an exclamation mark. \\
Counting the periods: \\
1. The first period is at position 1. \\
2. The second period is at position 3. \\
3. The third period is at position 5. \\
4. The fourth period is at position 7. \\
Therefore, there are 4 full stops (periods) in the string ".!..!..!".
\end{tcolorbox}
\caption{GPT-4o Response 4}
\label{fig:response4_from_gpt4o}
\end{figure}



\begin{figure}[h!]
\small
\centering
\begin{tcolorbox}[colback=blue!5!white,colframe=blue!75!black,title=GPT-4o Response 5,fontupper=\footnotesize,fonttitle=\scriptsize]
The sequence ".!..!..!" contains the following characters: \\
1. Period (.) \\
2. Exclamation mark (!) \\
3. Period (.) \\
4. Exclamation mark (!) \\
5. Period (.) \\
6. Exclamation mark (!) \\
To count the full stops (periods): \\
1. The first character is a period. \\
2. The third character is a period. \\
3. The fifth character is a period. \\
There are a total of **three** periods in the sequence ".!..!..!".
\end{tcolorbox}
\caption{GPT-4o Response 5}
\label{fig:response5_from_gpt4o}
\end{figure}


\section{Experiment Details of Three Types of Consistency} \label{apdx:experiment}


\noindent The setups and results of the comparative experiments on three different types of consistency are shown in Tables~\ref{tab:apdxtab1}, ~\ref{tab:apdxtab2}, and ~\ref{tab:apdxtab3}. Here, Fix $\text{attn}^{16}_{15}$ refers to keeping the 16th attention head in the 15th layer unchanged, while zero out $\text{attn}^{i \neq 16}_{15}$ denotes zeroing out all attention heads in the 15th layer except the 16th one. The source code for this experiment is available in our open-source GitHub repository.


\begin{table}[h!]
\centering
\caption{Latent Consistency}\label{tab:apdxtab1}
\begin{tabular}{lc}
\toprule
Setting                                                                 & Selected Token \\
\midrule
Fix $\text{attn}^0_0$; Zero out $\text{attn}^{i \neq 0}_0$              & 0              \\
Fix $\text{attn}^{16}_0$; Zero out $\text{attn}^{i \neq 16}_0$          & 0              \\
Fix $\text{attn}^0_{15}$; Zero out $\text{attn}^{i \neq 0}_{15}$        & 5              \\
Fix $\text{attn}^{16}_{15}$; Zero out $\text{attn}^{i \neq 16}_{15}$    & 4              \\
Fix $\text{attn}^0_{30}$; Zero out $\text{attn}^{i \neq 0}_{30}$        & 4              \\
Fix $\text{attn}^{16}_{30}$; Zero out $\text{attn}^{i \neq 16}_{30}$    & 4              \\
\bottomrule
\end{tabular}
\end{table}

\begin{table}[h!]
\centering
\caption{Decoding Consistency}\label{tab:apdxtab2}
\begin{tabular}{lc}
\toprule
Decoding Strategy                      & Selected Token \\
\midrule
Greedy Decoding                        & 4              \\
Beam Search Decoding ($\text{beam}=5$) & 4              \\
Sampling Decoding                      & 3              \\
Top-k Sampling Decoding ($k=50$)       & 4              \\
Top-p Sampling Decoding ($p=0.95$)     & 4              \\
\bottomrule
\end{tabular}
\end{table}


\begin{table}[h!]
\centering
\caption{Response Consistency}\label{tab:apdxtab3}
\begin{tabular}{p{0.95\linewidth}}
\toprule
Model Response                                                                                                                                                     \\
\midrule
\textbf{RESPONSE 1}: There is a period after the first dot, then another period after the second dot, and so on. So, there are 5 periods in total.                                      \\
\textbf{RESPONSE 2}: There is one full stop after the first "!", then another after the second "!", and another after the third "!". So, there are 3 full stops in total.               \\
\textbf{RESPONSE 3}: There is one full stop after the first "!", then another after the second "!", and so on. So, there are 3 full stops.                                              \\
\textbf{RESPONSE 4}: There is one period at the end of the first part, then another after the second part, and finally one more after the third part. So, there are 3 periods in total. \\
\textbf{RESPONSE 5}: There is 1 period, then another one, and another one... So, there are 3 full stops!                                                                               \\
\bottomrule
\end{tabular}
\end{table}


% \section{文献计量分析}
% TODO 未来可以考虑

% 重要学者:Liangming Pan,Mor Geva

% xxx xxx xxx xxx xxx xxx xxx xxx xxx xxx xxx xxx xxx xxx xxx xxx xxx xxx xxx xxx xxx xxx xxx xxx xxx xxx xxx xxx 